信道解码使用维特比译码器,判决方式为硬判决。设计思路如下:

\paragraph{变量设计}

\begin{itemize}
\item \textbf{codebuff} : 接收的码流,存于缓冲区中,数量达到一组后送入译码区
\item \textbf{code} : 译码区码流,即当然需要译码的码流
\item \textbf{possible\_codex} : 维特比译码图中,当前阶段第x个状态(共有四个状态)所对应的最优路径
\item \textbf{errorx} : 第x个最优路径(共有4 条路径)对应的判决误差
\item \textbf{decodex} : 维特比译码中,当前阶段第x状态的最优路径所对应的输出译码
\item \textbf{decode} : 前一组码流(已译码结束)所对应的输出译码
\end{itemize}

\paragraph{译码流程}

\begin{enumerate}
\item 接收码流至codebuff,当接收完一组后,送入code中等待译码
\item 每一个时钟周期,译码往前推进两位,求出当前状态下的最优路径并计算硬判决误差,并记录四个状态所对应的最优路径
\item 完成一组译码后,将译码结果存入decode,之后每一个时钟周期输出一位译码结果
\end{enumerate}