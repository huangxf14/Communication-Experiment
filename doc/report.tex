\documentclass[UTF8]{ctexart}
\usepackage{booktabs}  % professionally typeset tables
\usepackage{amsmath}
\usepackage{setspace}
\usepackage{textcomp}  % better copyright sign, among other things
\usepackage{xcolor}
\usepackage{lipsum}    % filler text
\usepackage{subfig}   
\usepackage{geometry}
\usepackage{float}
\usepackage{hyperref}
\usepackage{graphicx}


\geometry{left=2.54cm,right=2.54cm,top=2.18cm,bottom=3.18cm}
\date{}
\title{\textbf{通信原理实验}}
\author{无45 \ *** \ **********\\
        无45 \ *** \ **********\\
        无46 \ 黄秀峰 \ 2014011193\\
        无47 \ *** \ **********}


\begin{document}
\maketitle

\section{实验目的}

\section{实验平台}

\section{实验内容}

\subsection{系统功能}

\subsection{系统结构}

\subsubsection{信道编码模块}

\subsubsection{交织模块}

\subsubsection{调制模块}

\subsubsection{信道模拟模块}

\subsubsection{解调模块}

\subsubsection{解交织模块}

\subsubsection{信道解码模块}

信道解码使用维特比译码器,判决方式为硬判决。设计思路如下:

\paragraph{变量设计}

\begin{itemize}
\item \textbf{codebuff} : 接收的码流,存于缓冲区中,数量达到一组后送入译码区
\item \textbf{code} : 译码区码流,即当然需要译码的码流
\item \textbf{possible\_codex} : 维特比译码图中,当前阶段第x个状态(共有四个状态)所对应的最优路径
\item \textbf{errorx} : 第x个最优路径(共有4条路径)对应的判决误差
\item \textbf{decodex} : 维特比译码中,当前阶段第x状态的最优路径所对应的输出译码
\item \textbf{decode} : 前一组码流(已译码结束)所对应的输出译码
\end{itemize}

\paragraph{译码流程}

\begin{enumerate}
\item 接收码流至codebuff,当接收完一组后,送入code中等待译码
\item 每一个时钟周期,译码往前推进两位,求出当前状态下的最优路径并计算硬判决误差,并记录四个状态所对应的最优路径
\item 完成一组译码后,将译码结果存入decode,之后每一个时钟周期输出一位译码结果
\end{enumerate}

\subsection{软件仿真}

\section{实验总结}


\end{document}





