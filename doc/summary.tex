在本次实验中,我们设计了基于(2,1,3)卷积码和QPKS调制的通信系统,使用LED灯作为输出,在有噪声的情况下检验了该系统的性能。实验证明,卷积码在一定噪声情况下依然可以有较好的性能。

由于我们对于卷积码有着较为充分的了解,且团队成员中有人非常擅长编写Verilog程序,因此总体上实验进行的较为顺利。虽然如此,各模块之间的同步依然是一个比较困扰我们的问题。在最开始设计的时候我们没有充分考虑到模块同步性的问题,仿真中出现了颇令人费解的现象,在逐级排查解决问题之后,我们终于使得系统能够正常工作。不过此时在边界情况下仍然存在一些小问题,我们尝试了不同的输入情况,尽可能充分地解决了这些问题。

最后,感谢老师和助教们这一学期的悉心指导和辛苦付出,祝二零一八年一切顺利!\\
 
  
   
    
{\bf \Large 分工情况:}
\begin{itemize}
\item 黄天昊:信道调试/噪声模块编写,以及相应部分报告撰写。
\item 孙径舟:卷积编码,模块联调与测试,以及相应部分报告撰写。{\bf 软件仿真}与{\bf 总结}撰写。
\item 黄秀峰:卷积解码,模块联调与测试,以及相应部分报告撰写。{\bf 实验情况介绍}撰写。
\item 刘畅:交织/解交织模块编写,以及相应部分报告撰写。
\end{itemize}

